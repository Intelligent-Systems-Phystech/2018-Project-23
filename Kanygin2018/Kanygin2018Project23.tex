\documentclass[12pt,twoside]{article}
\usepackage{jmlda}
%\NOREVIEWERNOTES
\title
    [Фрактальный анализ и синтез оптических изображений морского волнения] % Краткое название; не нужно, если полное название влезает в~колонтитул
    {Фрактальный анализ и синтез оптических изображений морского волнения}
\author
    [Автор~И.\,О.] % список авторов для колонтитула; не нужен, если основной список влезает в колонтитул
    {Каныгин~Ю.\,Ю., Лукошин~В.\,О., Фамилия~И.\,О.} % основной список авторов, выводимый в оглавление
    [Каныгин~Ю.\,Ю., Лукошин~В.\,О.$^2$, Фамилия~И.\,О.$^2$] % список авторов, выводимый в заголовок; не нужен, если он не отличается от основного
\thanks
    {Работа выполнена при финансовой поддержке РФФИ, проект \No\,00-00-00000.
   Научный руководитель:  Стрижов~В.\,В.
   Задачу поставил:  Матвеев~И.\,О.
    Консультант:  Консультант~И.\,О.}
\email
    {author@site.ru}
\organization
    {$^1$Московский физико-технический институт (Государственный Университет); $^2$Организация}
\abstract
    {Данная работа посвящена исследованию численными методами зависимости между характеристиками пространственных спектров морских волн и фрактальной размерностью спутниковых изображений Земли в области солнечного блика.

\bigskip
\textbf{Ключевые слова}: \emph {ключевое слово, ключевое слово,
еще ключевые слова}.}
\titleEng
    {Название на заморском}
\authorEng
    {Author~F.\,S.$^1$, CoAuthor~F.\,S.$^2$, Name~F.\,S.$^2$}
\organizationEng
    {$^1$Moscow Institute of Physics and Technology (State University); $^2$Organization}
\abstractEng
    {This document is ....

    \bigskip
    \textbf{Keywords}: \emph{keyword, keyword, more keywords}.}
\begin{document}
\maketitle
%\linenumbers
\section{Введение}
После аннотации, но перед первым разделом,
располагается введение, включающее в себя
описание предметной области,
обоснование актуальности задачи,
краткий обзор известных результатов,
и~т.\,п~\cite{author09anyscience,myHandbook,author09first-word-of-the-title,voron06latex,author-and-co2007,Lvovsky03}.



\section{Некоторые формулы}

Образец формулы: $f(x_i,\alpha^\gamma)$.

Образец выключной формулы без номера:
\[
    y(x,\alpha) =
    \begin{cases}
        -1, & \text{если } f(x,\alpha)<0;  \\
        +1, & \text{если } f(x,\alpha)\geq 0.
    \end{cases}
\]

Образец выключной формулы с номером:
\begin{equation}
\label{eq:cases}
    y(x,\alpha) =
    \begin{cases}
        -1, & \text{если } f(x,\alpha)<0;  \\
        +1, & \text{если } f(x,\alpha)\geq 0.
    \end{cases}
\end{equation}

Образец выключной формулы, разбитой на две строки с~помощью окружения align:
\begin{align}
    R'_N(F)
        = \frac1N \sum_{i=1}^N
        \Bigl(
            & P(+1\cond x_i) C\bigl(+1,F(x_i)\bigr)+{}
        \notag % подавили номер у первой строки
    \\ {}+{}
            & P(-1\cond x_i) C\bigl(-1,F(x_i)\bigr)
        \Bigr).
        \label{eq:R(F)}
\end{align}

Образцы ссылок: формулы~\eqref{eq:cases} и~\eqref{eq:R(F)}.


\section{Заключение}
Желательно, чтобы этот раздел был, причём он не~должен дословно повторять аннотацию.
Обычно здесь отмечают,
каких результатов удалось добиться,
какие проблемы остались открытыми.

\newpage
%\bibliographystyle{unsrt}
%\bibliography{jmlda-bib}
\begin{thebibliography}{1}

\bibitem{author09anyscience}
    \BibAuthor{Лупян\;Е.}
    \BibTitle{Возможности фрактального анализа оптических изображений морской поверхности}~//
    \BibJournal{10-th Int'l. Conf. on Anyscience}, 2009.  Vol.\,11, No.\,1.  Pp.\,111--122.
\bibitem{myHandbook}
    \BibAuthor{Бондур\;В.\,Г.}
    Методы восстановления спектров морского волнения по спектрам аэрокосмических изображений.
    Город: Издательство, 2009. 314~с.
\bibitem{author09first-word-of-the-title}
    \BibAuthor{Автор\;И.\,О.}
    \BibTitle{Название статьи}~//
    \BibJournal{Название конференции или сборника},
    Город:~Изд-во, 2009.  С.\,5--6.
\bibitem{author-and-co2007}
    \BibAuthor{Автор\;И.\,О., Соавтор\;И.\,О.}
    \BibTitle{Название статьи}~//
    \BibJournal{Название журнала}. 2007. Т.\,38, \No\,5. С.\,54--62.
\bibitem{bibUsefulUrl}
    \BibUrl{www.site.ru}~---
    Название сайта.  2007.
\bibitem{voron06latex}
    \BibAuthor{Воронцов~К.\,В.}
    \LaTeXe\ в~примерах.
    2006.
    \BibUrl{http://www.ccas.ru/voron/latex.html}.
\bibitem{Lvovsky03}
    \BibAuthor{Львовский~С.\,М.} Набор и вёрстка в пакете~\LaTeX.
    3-е издание.
    Москва:~МЦHМО, 2003.  448~с.
\end{thebibliography}

% Решение Программного Комитета:
%\ACCEPTNOTE
%\AMENDNOTE
%\REJECTNOTE
\end{document}
